\documentclass{article}
\usepackage{graphicx} 
\usepackage[T2A]{fontenc}
\usepackage{newtxtext, newtxmath}

\title{Дизайн документ}
\author{Надежда Мельникова, Валентина Дашинимаева, Наталья Хлопочкина, \\
Степан Васильченко, Никита Маряхин, Александр Фурман}
\date{Ноябрь 2024}

\begin{document}

\maketitle
\renewcommand{\contentsname}{Оглавление}
\tableofcontents

\section{Введение}

\section{Концепция}

\subsection{Введение}

\subsection{Жанр и аудитория}

\subsection{Основные особенности игры}

\subsection{Описание игры}

\subsection{Предпосылки создания}

\subsection{Платформа}

\section{Функциональная спецификация}

\subsection{Принципы игры}

\subsubsection{Суть игрового процесса}

\subsubsection{Ход игры и сюжет}

\subsection{Физическая модель}

\subsection{Персонаж игрока}

\subsection{Элементы игры}

\subsection{«Искусственный интеллект»}

\subsection{Многопользовательский режим}

\subsection{Интерфейс пользователя}

\subsubsection{Блок-схема}

\subsubsection{Функциональное описание и управление}

\subsubsection{Объекты интерфейса пользователя}

\subsection{Графика и видео}

\subsubsection{Общее описание}

\subsubsection{Двумерная графика и анимация}

\subsubsection{Трехмерная графика и анимация}

\subsubsection{Анимационные вставки}

\subsection{Звуки и музыка}

\subsubsection{Общее описание}

\subsubsection{Звук и звуковые эффекты}

\subsubsection{Музыка}

\subsection{Описание уровней}

\subsubsection{Общее описание дизайна уровней}

\subsubsection{Диаграмма взаимного расположения уровней}

\subsubsection{График введения новых объектов}

\section{Контакты}

\newpage

\end{document}

