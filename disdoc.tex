\documentclass{article}
\usepackage{graphicx} 
\usepackage[T2A]{fontenc}
\usepackage{newtxtext, newtxmath}
\usepackage[utf8]{inputenc}
\usepackage{geometry}
\geometry{top=2cm, bottom=2cm, left=2.5cm, right=2.5cm}
\usepackage{array}

\title{Дизайн документ}
\author{Надежда Мельникова, Валентина Дашинимаева, Наталья Хлопочкина, \\
Степан Васильченко, Никита Маряхин, Александр Фурман}
\date{Ноябрь 2024}

\begin{document}

\maketitle
\renewcommand{\contentsname}{Оглавление}
\tableofcontents

\section{Введение}

\ \ \ \ \ Здесь представлена информация, необходимая для понимания данного документа, посвященного разработке игры в жанре киберпанк, шутер и файтинг, с возрастным ограничением 16+ / 18+.

\subsection{Комментарии по организации содержимого документа}
\ \ \ \ \ Документ разделен на несколько ключевых разделов, включая концепцию игры, персонажей, игровой мир, механики и стилистику. Каждый раздел подробно описывает центральные элементы игры и будет служить основой для дальнейшей разработки.


\subsection{Список авторов}
\begin{itemize}
    \item Надежда Мельникова 
    \item Валентина Дашинимаева 
    \item Наталья Хлопочкина
    \item Степан Васильченко 
    \item Никита Маряхин 
    \item Александр Фурман 
\end{itemize}


\subsection{Дополнительные сведения}
\ \ \ \ \ Данный документ предназначен для команды разработки и заинтересованных сторон. Знание контекста игры, ее тематики и целевой аудитории является важным для обсуждений и успешной реализации идеи. Игра устанавливает возрастные ограничения 16+ / 18+, что подразумевает, что все элементы, такие как сюжетная линия, визуальные и аудио материалы, должны полностью соответствовать этим критериям.

\section{Концепция}

\subsection{Введение}
\ \ \ \ \ В старом городе ты играешь за закаленного жизнью маргинала, который уже привык к жизни в трущобах и катакомбах. В мире, где импланты — обыденность, а замена конечностей — рядовая хирургическая процедура, твои механические руки — единственное, что осталось от прежней жизни. Проснувшись в жуткой хирургической палате с сильной головной болью и пустым местом, где прежде были твои протезы, ты понимаешь — тебя ограбили. Начинается твое путешествие в поисках ответа на вопрос: кто посмел лишить тебя самого ценного? И не только ответа, но и жестокой мести. Готовься к опасным приключениям, раскрытию темного прошлого героя и нелёгкому пути к справедливости в этом суровом мире.
\subsection{Жанр и аудитория}
\ \ \ \ \ Наша игра будет совмещать в себе несколько жанров: киберпанк, шутер, файтинг, платформер и Hack and slash.

Игра направлена на людей, которым нравятся игры с  мрачной атмосферой и динамичными боевыми действиями с использованием разнообразного оружия и тактик. 

Возрастное ограничение: 18+
\subsection{Основные особенности игры}

\begin{itemize}
    \item Ближний бой включает использование подручных предметов и тактические приемы, которые варьируются в зависимости от стиля врага.
    \item Битвы сочетают динамичность и мгновенную смертельную опасность, вдохновляясь механиками Hotline Miami и Katana Zero, что делает каждую схватку испытанием на скорость реакции и стратегическое мышление.
    \item Мрачный, многослойный мир и нелинейный сюжет. Антиутопический город разделен на визуально и тематически отличающиеся районы. Игрок сможет раскрывать тайны этого мира, влияя на сюжет своими решениями, что добавляет реиграбельности и глубины повествованию.
    \item Игроку предлагается свобода выбора стиля игры: от скрытного прохождения с минимальным использованием силы до яростных сражений. Этот гибридный подход повышает реиграбельность.
    \item Возможность скрытного прохождения уровней подчеркивает разнообразие тактик, позволяя выбрать между прямым столкновением и осторожным устранением целей.
    \item В отличие от Cyberpunk 2077, игра делает акцент не на масштабе открытого мира, а на его плотности и проработке деталей.
    \item Вдохновляясь Hotline Miami, проект добавляет глубину через исследование и сюжетные выборы, чего не хватает в большинстве шутеров с быстрой боевой механикой.
    \item Использование темы утраты механических рук выделяет героя среди клише "суперсильных" протагонистов и заставляет игрока чувствовать себя уязвимым.
\end{itemize}

\subsection{Описание игры}
\ \ \ \ \ Погрузитесь в мрачный и детально проработанный мир киберпанка, где технологии переплетаются с повседневной жизнью до неузнаваемости. Ваша история начинается в разрушенном старом городе, пострадавшем от масштабной технической революции. Главный герой — житель трущоб и катакомб, привыкший к маргинальной жизни на самом дне социума. В этом мире операции по замене частей тела на искусственные импланты стали нормой, но дорогостоящей роскошью. Единственная ценность, которой по-прежнему дорожит герой — его механические руки.

\subsubsection{Сюжетная линия}

\ \ \ \ \ История начинается с неожиданного пробуждения главного героя в странной и мрачной хирургической палате. Голова раскалывает боль, а уши звонят после вчерашнего вечера, проведенного в запое. Обычно герой не удивляется, просыпаясь в незнакомых местах, но на этот раз что-то не так: его ценные механические протезы исчезли. Теперь перед ним стоит задача не только выжить, но и отомстить тем, кто лишил его части тела и, возможно, личности.

\subsubsection{Игровой процесс}

\ \ \ \ \ Игроку предстоит управлять героем, используя сочетание традиционного огнестрельного оружия и боевых навыков, а также доступных имплантов для преодоления препятствий и сражений. Игра сочетает элементы шутера и файтинга, предоставляя динамичные и напряженные бои с разнообразными противниками. В процессе прохождения игрок будет:
 • Исследовать мрачные районы города: От заброшенных трущоб до высокотехнологичных комплексов, каждый район полон секретов и опасностей;
 • Бороться с врагами: Использовать различные виды оружия и боевых приемов, а также улучшать свои импланты для повышения боевых возможностей;
 • Развивать персонажа: Собирая ресурсы и выполняя задания, игрок сможет улучшать механические руки и приобретать новые способности;
 • Взаимодействовать с NPC: Узнавать больше о прошлом героя, заводить союзников и противостоять врагам, раскрывая глубокие сюжетные линии;
 • Принимать решения: Ваши выборы будут влиять на развитие сюжета и концовку игры, создавая уникальный опыт прохождения.

\subsubsection{Визуальный и звуковой стиль}

\ \ \ \ \ Игра вдохновлена такими проектами, как Katana Zero, Hotline Miami и Cyberpunk 2077. Визуальный стиль сочетает неоновую палитру киберпанка с мрачными, детализированными локациями, создавая атмосферу напряженности и неизведанности. Саундтрек включает синтезаторные композиции и динамичные треки, усиливающие эмоциональное воздействие и погружение в игровой мир.




\subsection{Предпосылки создания}
\ \ \ \ \ Игра опирается на актуальные тенденции рынка, где киберпанковая эстетика и глубокий нарратив продолжают привлекать внимание игроков. 
После успеха таких проектов, как Cyberpunk 2077, Deus Ex и Ghostrunner, очевидно, что жанр киберпанк сохраняет популярность.
Игра стремится предложить уникальный взгляд на жанр, акцентируя внимание на психологических аспектах и социальном неравенстве, 
что делает её интересной для широкой аудитории.

Рынок также демонстрирует рост популярности игр с динамичными механиками боя и нелинейным повествованием. По данным аналитических отчётов 
в 2023 году игры с подобными характеристиками показывали устойчивый рост продаж, особенно на платформах ПК и консолях. Это подтверждает, 
что игра имеет потенциал занять достойное место среди успешных проектов.

Вопросы лицензирования в разработке были заранее учтены. Мы используем лицензированный движок Unreal Engine, что обеспечивает 
высокую производительность и визуальную привлекательность игры. Кроме того, весь музыкальный и графический контент либо создаётся 
собственными силами, либо приобретается через лицензированные платформы (например, Unity Asset Store или AudioJungle), что гарантирует 
отсутствие проблем с авторскими правами.

\subsection{Платформа}

Игра разрабатывается для следующих платформ:
\begin{itemize}
    \item PC (Windows, macOS, Linux)
    \item PlayStation
    \item Xbox
    \item Nintendo Switch
\end{itemize}

Для PC-версии игры минимальные и рекомендуемые системные требования приведены ниже:

\begin{table}[h!]
\centering
\renewcommand{\arraystretch}{1.5}
\begin{tabular}{|l|l|l|}
\hline
\textbf{Требования}           & \textbf{Минимальные}                        & \textbf{Рекомендуемые}                    \\ \hline
Операционная система          & Windows 10, Ubuntu 20.04                    & Windows 11, macOS Ventura                 \\ \hline
Процессор                     & Intel Core i3-6100 / AMD FX-6300            & Intel Core i5-10400 / AMD Ryzen 5 3600    \\ \hline
ОЗУ                           & 8 ГБ                                        & 16 ГБ                                     \\ \hline
CD-ROM привод                 & Не требуется                               & Не требуется                             \\ \hline
Свободное место на HDD        & 20 ГБ                                       & 50 ГБ (SSD рекомендуется)                \\ \hline
Видеокарта                    & NVIDIA GTX 750 Ti / AMD R7 260X             & NVIDIA GTX 1660 / AMD RX 580             \\ \hline
Звуковая карта                & Любая совместимая с DirectX                 & Высококачественная                       \\ \hline
Управление                    & Клавиатура и мышь                           & Геймпад (рекомендуется)                  \\ \hline
\end{tabular}
\caption{Системные требования для PC}
\end{table}

Дополнительное оборудование: Не требуется.

\section{Функциональная спецификация}

\subsection{Принципы игры}

\subsubsection{Суть игрового процесса}

Игровой процесс (gameplay) игры предлагает сочетание насыщенного сюжета, динамичных сражений, исследований и персонализации героя, что в совокупности создаёт увлекательный опыт. Основные аспекты игрового процесса включают:

Основные элементы игрового процесса
\begin{enumerate}
    \item \textbf{Исследование мира:}  
    Игрок сможет исследовать большой мрачный мегаполис, разбитый на несколько уникальных районов:
    \begin{itemize}
        \item Разрушенные трущобы с граффити и самодельными ловушками.
        \item Заброшенные подземные катакомбы, где можно найти редкие ресурсы.
        \item Высокотехнологичные районы, охраняемые частными военными корпорациями.
    \end{itemize}
    Исследование мира дарит ощущение загадки и свободы, заставляя искать скрытые проходы, секреты и подсказки.
    
    \item \textbf{Динамичные сражения:}  
    Основная механика боя сочетает шутер и файтинг.
    \begin{itemize}
        \item Игрок сможет использовать огнестрельное оружие: пистолеты, дробовики, энергетические винтовки.
        \item Атаковать и защищаться с помощью своих имплантов (например, мощный удар механической рукой или защита от энергетических атак).
        \item Враги варьируются от обычных уличных бандитов до элитных наёмников корпораций и механических роботов.
    \end{itemize}
    Боевая система дарит удовольствие за счёт напряжённых, динамичных и визуально эффектных битв.
    
    \item \textbf{Прокачка персонажа:}  
    \begin{itemize}
        \item Игрок собирает ресурсы, необходимые для улучшения механических частей тела, таких как руки, ноги или глазные импланты.
        \item Модификации позволяют повысить боевую мощь, например, увеличить силу удара, повысить скорость или добавить специальные навыки, такие как взлом электроники или невидимость.
    \end{itemize}
    Удовольствие заключается в постепенном усилении героя и раскрытии новых возможностей.
    
    \item \textbf{Взаимодействие с NPC:}  
    \begin{itemize}
        \item Игрок будет встречать харизматичных персонажей с уникальными историями и характерами.
        \item Диалоги могут раскрыть тайны мира, помочь герою в развитии или открыть доступ к скрытым миссиям.
        \item Некоторые NPC могут стать союзниками в его миссии мести, а другие – предателями.
    \end{itemize}
    \item \textbf{Побочные миссии:}  
    Игра полна скрытых зон, необязательных квестов и пасхалок, которые могут дать редкие улучшения или углубить сюжет.  
    Исследование мира и выполнение дополнительных задач дарят игроку чувство открытия и награды.
\end{enumerate}

\subsubsection{Ход игры и сюжет}
Катастрофический сбой системы угрожает разрушить нейронную сеть города, затронув всех, 
кто обладает кибернетическими усовершенствованиями. Персонаж игрока, «кибер-механик» со специальными навыками, 
обладает уникальными способностями для устранения проблемы. Однако он быстро обнаруживает, 
что сбой не был случайным, а является преднамеренным актом саботажа. 
Игрок сталкивается с культом "Чистота",
который стремится искоренить саму концепцию кибернетики. Культ вскоре связывается с главным героем, объясняя,
что кибернетика не только обесчеловечивает людей, но и является инструментом влияния. По их словам импланты 
влияют на разум носителя, тем самым после длительного ношения и вовсе превращают в марионетку того самого ИИ.
В дальнейшем игроку предстоит выяснить правда ли это и принять решение на чью сторону встать. Вследствие своего расследования
Игрок узнает, что часть истины в словах культа имеется. Со временем каждый носитель мозговых имплантов действительно теряет волю,
но вместе с тем становится известно, что при уничтожении ИИ все, кто носит такие импланты, погибнут.
Перед игроком встает выбор уничтожить ИИ и избавиться от самой крупной системы массового контроля, но дать погибнуть миллионам людей
или примкнуть к системе и уничтожить культ "Чистота".


\subsection{Физическая модель}

\subsubsection{Перемещения}

\paragraph{Общие принципы движения.}
\begin{itemize}
    \item \textbf{Перемещение персонажа:} Модель базируется на законах классической механики Ньютона, учитывая массу, ускорение и трение:
    \[
    F = ma, \quad a = \frac{F}{m}
    \]
    где \(F\) — сила, \(m\) — масса персонажа (включая имплантаты), \(a\) — ускорение. 
    \item \textbf{Прыжки:} Высота прыжка определяется импульсом, созданным ногами персонажа:
    \[
    h = \frac{v^2}{2g}
    \]
    где \(v\) — начальная скорость, \(g\) — ускорение свободного падения.
    \item \textbf{Скольжение и уклонение:} Уровень трения варьируется в зависимости от поверхности (например, скользкий пол в лабораториях или грубая текстура разрушенных улиц). Трение рассчитывается как:
    \[
    F_\text{тр} = \mu N
    \]
    где \(\mu\) — коэффициент трения, \(N\) — нормальная сила.
\end{itemize}

\subsubsection{Боевые действия}

\paragraph{Механика ближнего боя.}
\begin{itemize}
    \item \textbf{Сила удара:} Рассчитывается с учётом массы механических конечностей и скорости их движения:
    \[
    F_\text{уд} = m_\text{руки} a_\text{руки}
    \]
    где \(m_\text{руки}\) — масса кибернетической руки, \(a_\text{руки}\) — её ускорение.
    \item \textbf{Повреждения:} Урон зависит от кинетической энергии удара:
    \[
    E_\text{к} = \frac{1}{2} m_\text{руки} v^2
    \]
    где \(v\) — скорость удара.
\end{itemize}

\paragraph{Механика дальнего боя.}
\begin{itemize}
    \item \textbf{Полет снарядов:} Траектория учитывает гравитацию и сопротивление воздуха (упрощённо):
    \[
    y = y_0 + v_{y0} t - \frac{1}{2} g t^2
    \]
    где \(y_0\) — начальная высота, \(v_{y0}\) — вертикальная составляющая скорости, \(t\) — время, \(g\) — ускорение свободного падения.
    \item \textbf{Энергетическое оружие:} Лучи игнорируют сопротивление воздуха, но имеют ограниченный радиус действия и мощность, рассчитываемую по формуле:
    \[
    P = \frac{E}{t}
    \]
    где \(P\) — мощность, \(E\) — энергия луча, \(t\) — время воздействия.
\end{itemize}

\subsection{Персонаж игрока}

\textbf{Аватар игрока: Кибер-механик}

Игрок берет на себя роль \textbf{Кибер-механика} — талантливого инженера, специализирующегося на ремонте и модификации кибернетических систем. Герой имеет глубокую связь с технологией благодаря собственным усовершенствованным имплантатам, которые не только поддерживают его жизнь, но и делают его уникальным в своей области. 

\paragraph{Внешний вид персонажа.}
Кибер-механик — это сочетание человека и машины. Его облик варьируется в зависимости от выбора игрока, но ключевые элементы включают:
\begin{itemize}
    \item \textbf{Механические руки:} усиленные конечности с модульной конструкцией, позволяющие быстро адаптироваться к ситуации. Они могут быть оснащены инструментами для ремонта, оружием или интерфейсами для взлома систем.
    \item \textbf{Кибернетические глаза:} оснащены функциями увеличения, теплового зрения и интерфейса дополненной реальности, помогающими ориентироваться в сложных условиях.
    \item \textbf{Гибридный костюм:} защитная броня с интегрированными батареями и интерфейсами, обеспечивающая как защиту, так и поддержку в бою.
    \item \textbf{Шрамы и повреждения:} на коже видны следы от предыдущих операций, аварий или боев, подчёркивающие суровый характер мира.
\end{itemize}

\paragraph{Характеристика героя.}
\begin{itemize}
    \item \textbf{Технические навыки:} Персонаж обладает способностью быстро взламывать сложные системы, ремонтировать устройства на ходу и адаптировать технологии для своего преимущества.
    \item \textbf{Боевая подготовка:} Кибер-механик комбинирует навыки ближнего боя (с использованием своих мощных имплантатов) и дальнего боя, используя разнообразное оружие.
    \item \textbf{Моральная дилемма:} Игрок сталкивается с выбором между восстановлением порядка в городе или использованию хаоса для собственных целей. Это отражается в диалогах и ключевых сюжетных решениях.
\end{itemize}

\paragraph{Механика кастомизации.}
Игроку предоставляется возможность персонализировать своего персонажа. Это касается не только внешнего вида (выбор кибернетических модулей и их дизайна), но и характеристик:
\begin{itemize}
    \item \textbf{Выбор модификаций:} Различные улучшения дают уникальные способности, например, невидимость, увеличение силы удара или улучшение взлома.
    \item \textbf{Развитие навыков:} Игрок может фокусироваться на боевых, технических или дипломатических умениях.
    \item \textbf{Влияние решений:} Уникальные взаимодействия с NPC и выбор пути влияют на репутацию героя, открывая новые возможности или ограничивая доступ к некоторым ресурсам.
\end{itemize}

\paragraph{Мотивация персонажа.}
Кибер-механик был втянут в кризис из-за своих уникальных способностей. Его первичная цель — восстановить городскую нейронную сеть, чтобы предотвратить катастрофу. Однако, по мере раскрытия заговора, перед героем встают более сложные задачи: определить, кто является настоящим врагом, и решить, стоит ли полностью восстановить старую систему или создать что-то новое. Личное прошлое персонажа, связанное с потерей близких в результате корпоративных игр, придает его действиям эмоциональную глубину.

Персонаж игрока является ядром повествования, и его развитие неразрывно связано с выбором игрока, делая каждого прохождение уникальным.

\subsection{Элементы игры}

В этом разделе описываются основные элементы игры, которые формируют её уникальный игровой процесс и взаимодействие с игроком.
\subsubsection{Персонажи (Юниты)}
Главный герой - это игрок, который оказывается в трудной ситуации. Он имеет механические руки, которые не только являются его ценностью, но также позволяют ему использовать специальные способности и уникальные приемы в бою.
\begin{itemize}
   \item \textbf{Жизни (Health)}: Параметр, отвечающий за выживание героя. При снижении этого параметра герой погибает и игру придется начать заново.
    \item \textbf{Энергия (Energy)}: Используется для активации специальных приёмов, связанных с механическими протезами. Энергия восстанавливается со временем или при помощи специальных предметов.
    \item \textbf{Умения (Skills)}: Уникальные навыки героя, включая боевые приёмы с использованием механических рук и специальных укрытий. Игрок сможет развивать эти навыки по мере прохождения игры.

\end{itemize}
Различные враги населяют игровой мир.

\begin{itemize} 
     \item \textbf{Типы врагов:} Обычные бандиты, мутировавшие люди, кибернетические охранники, роботы, боссы. 
     \item \textbf{Характеристики врагов:} Здоровье (Здоровье), Урон (Урон), Скорость (Скорость), Броня (Броня - опционально), Уязвимости (Слабости). Более сильные враги обладают специальными атаками или способностями.
\end{itemize}
\subsubsection{Оружие}
В игре представлено разнообразие оружия, которое может быть использовано в бою:
\begin{itemize}
    \item \textbf{Механические руки}: Главный герой использует свои механические руки как оружие, что позволяет ему наносить высокие повреждения противникам в ближнем бою.
    \item \textbf{Оружие на дальних дистанциях}: Игрок может находить разнообразные огнестрельные и энергетические пистолеты и автоматы, помогающие вести бой на расстоянии.
    \item \textbf{Специальные боеприпасы}: В игре будут доступны различные типы патронов, которые могут иметь разные эффекты, например, замедление врагов или поджог.
\end{itemize}
\subsubsection{Предметы}
Игрок сможет подбирать и использовать различные предметы, которые окажут влияние на его развитие и геймплей:
\begin{itemize}
    \item \textbf{Аптечки (Medkits)}: Позволяют восстанавливать жизнь. Будут как базовые, так и особые, восстанавливающие большее количество здоровья.
    \item \textbf{Модификации (Upgrades)}: Игрок может находить запчасти для улучшения своих механических рук и оружия, увеличивая их характеристики.
    \item \textbf{Временные бустеры (Boosters)}: Временные усиления, которые могут увеличивать скорость, урон или восстановление энергии на определенный период времени.
\end{itemize}

\subsubsection{Настройки и окружающая среда}
Игровая среда будет включать в себя:
\begin{itemize}
    \item \textbf{Трущобы и катакомбы}: Главные локации, в которых развернётся действие игры. 
    
    Атмосфера: мрачная и присутствует ощущение постоянной опасности.

    Опасности: обитают разные банды, встречаются ловушки, завалы и обломки зданий.

    Возможности: скрытые проходы, секретные укрытия, места для засада, возможность найти полезные ресурсы и редкие предметы.

    Визуальные особенности: темные, узкие улочки, заброшенные здания, груды мусора, граффити, поврежденные автомобили.
    \item \textbf{Высокотехнологичные районы (Зоны высоких технологий)}:  
    Атмосфера: холодная. 

    Опасности: вооруженные охранники, роботы, лазерные сетки, патрули, камеры слежения.

    Визуальные особенности: современные здания из стали и стекла, высокотехнологичное оборудование, светящиеся экраны, роботы, электронные замки.
\end{itemize}
\subsubsection{NPC}

Неигровые персонажи влияют на сюжет и игровой процесс.

\begin{itemize} 
    \item \textbf{Торговцы (Продавцы)}: 
    
    Чернорыночный торговец (Black Market Vendor): Этот торговец занимается нелегальными и редкими товарами, предлагая оружие, модификации и другие предметы по завышенным ценам. Их местоположение постоянно меняется, чтобы избежать обнаружения. 

    Механик (Mechanic): Ремонтирует оружие и механические руки за ресурсы, продает запчасти

\subsection{«Искусственный интеллект»}
Искусственный интеллект (AI) в игре играет важную роль в создании динамичных и сложных боевых ситуаций. AI контролирует поведение как врагов, так и некоторых союзников, влияя на общую атмосферу и опыт игрока.
\subsection{Основные принципы поведения AI:}

\begin{itemize}
    \item \textbf{Адаптивность:} AI врагов будет адаптироваться к действиям игрока. Если игрок предпочитает скрытное прохождение, враги будут искать его с помощью датчиков движения и камер. Если же игрок активно вступает в сражения, AI будет реагировать агрессивными контратаками, используя укрытия, стратегические маневры и комбинированные атаки. AI врагов будет учитывать тип оружия игрока и подбирать тактику, направленную на минимизацию угрозы.
    
    \item \textbf{Тактическое поведение:} Враги будут использовать окружающую среду для своего преимущества. Например, некоторые враги могут укрываться за объектами, другие могут использовать роботов или механических существ для выполнения атак. Важно, что AI врагов будет действовать не по заранее заданному сценарию, а в зависимости от ситуации, делая каждый бой уникальным.
    
    \item \textbf{Интеллектуальная поддержка союзников:} В игре также будет предусмотрен AI для союзников, который будет взаимодействовать с игроком, помогая ему в бою, предоставляя информацию и следуя за его действиями. AI союзников будет обучаться и адаптироваться к стилю игры игрока, предоставляя ему стратегическую помощь и создавая более захватывающую атмосферу.
\end{itemize}

\subsection{Многопользовательский режим}
Многопользовательский режим игры предоставляет игрокам возможность взаимодействовать друг с другом в реальном времени. Это расширяет игровой процесс и добавляет новые элементы стратегии, совместной работы и конкуренции.
\subsection{Описание многопользовательского режима:}

\begin{itemize}
    \item \textbf{Жанр:} Основной режим игры будет представлен в формате \textit{Deathmatch} (Бой на смерть).
    
    \item \textbf{Количество игроков:} Игроки смогут участвовать в сражениях в группе до \textit{10} человек. В зависимости от режима, количество игроков может варьироваться.
    
    \item \textbf{Характеристика действий игрока и доступных игровых элементов:} В отличие от одиночной кампании, в многопользовательском режиме игроки смогут использовать разнообразные тактики и команды. Каждый игрок может выбирать оружие, способности, а также может взаимодействовать с окружающей средой, создавать ловушки или использовать транспортные средства. Доступные элементы игры могут включать уникальные классы персонажей, улучшения и тактические возможности.
    
    \item \textbf{Организация сеансов и способ соединения:} Многопользовательский режим будет работать через выделенные серверы, с возможностью подключения игроков через интернет. Игроки смогут создавать свои игровые лобби или присоединяться к уже существующим сеансам. Игры будут проводиться в реальном времени, с возможностью настройки карт, правил и модификаций.
\end{itemize}

\clearpage
\subsection{Интерфейс пользователя}

\subsubsection{Блок-схема}
\begin{figure}[!htbp] 
    \centering 
    \includegraphics[width=0.5\textwidth]{diagram(1).png} 
    \caption{Блок-схема} 
\end{figure}

\subsubsection{Блок-схема}

\subsubsection{Функциональное описание и управление}

\subsection{Основные экраны игры}

\begin{itemize}
  \item \textbf{Главное меню}: Начать новую игру, загрузить сохранение, настройки, выход.
  \item \textbf{Экран загрузки}: Индикатор прогресса, подсказки.
  \item \textbf{Игровой экран}: Здоровье, боеприпасы, мини-карта, импланты, выносливость.
  \item \textbf{Меню паузы}: Продолжить, перезапустить уровень, настройки, выход в меню.
  \item \textbf{Экран инвентаря}: Управление оружием, имплантами, предметами.
\end{itemize}

\subsection{Элементы управления}

\textbf{ПК}:  
\begin{itemize}
  \item \texttt{W, A, S, D} — движение, \texttt{Левая кнопка мыши} — атака, \texttt{E} — взаимодействие, \texttt{Tab} — инвентарь.
  \item \texttt{Shift} — бег, \texttt{Esc} — пауза.
\end{itemize}

\textbf{Консоль}:  
\begin{itemize}
  \item Левый стик — движение, \texttt{Кнопка B} — уклонение, \texttt{Start} — пауза.
  \item \texttt{Y} — имплант, \texttt{X} — взаимодействие.
\end{itemize}

\subsection{Взаимодействие с игрой}

\begin{itemize}
  \item \textbf{Бой}: Огнестрельное оружие, боевые приемы, импланты для изменения течения боя.
  \item \textbf{Исследование}: Преодоление барьеров, поиск секретов с помощью имплантов.
  \item \textbf{Выборы}: Моральные решения, влияющие на сюжет.
\end{itemize}

\subsection{Адаптация интерфейса}

\begin{itemize}
  \item \textbf{Подсказки}: Оповещения о целях, достижениях.
  \item \textbf{Индикаторы}: Эффекты повреждений, перезарядки, использования имплантов.
\end{itemize}


\subsubsection{Объекты интерфейса пользователя}

\paragraph{Стандартные элементы интерфейса}
\begin{itemize}
    \item \textbf{Кнопка (Button)}: Кнопки могут использоваться для перехода между уровнями, активации особых режимов или открытия меню. Например, кнопка "Начать игру" запускает игровой процесс, кнопка "Пауза" останавливает игру, а кнопка "Выход" завершает сеанс.
    
    \item \textbf{Текстовое поле (Text field)}: Может использоваться для ввода имени игрока, пароля, или для отображения важных данных, таких как текущий уровень или количество очков.
    
    \item \textbf{Флажок (Checkbox)}: Может быть использован для включения или выключения опций в настройках игры, таких как звуковые эффекты, субтитры или предпочтения по графике (например, включение/выключение HDR).
    
    \item \textbf{Переключатель (Radio button)}: Используется для выбора одного из нескольких вариантов. Например, переключатель может быть использован в меню настроек для выбора уровня сложности игры (Легкий, Средний, Трудный).
    
    \item \textbf{Выпадающий список (Dropdown)}: В интерфейсе игры может быть использован для выбора конкретного персонажа, оружия или карты, например, в многопользовательском режиме.
\end{itemize}

\paragraph{Нестандартные элементы интерфейса}
\begin{itemize}
    \item \textbf{Интерактивные панели (Interactive panels)}: Например, панели могут отображать ресурсы или информацию о текущих заданиях, где игрок может кликать на различные элементы, чтобы быстро получить нужную информацию.
    
    \item \textbf{Иконки навыков/умений (Ability icons)}: В играх с боевыми механиками, иконки навыков играют важную роль в интерфейсе. Они часто требуют особого внимания, поскольку игрок может использовать их в реальном времени во время игрового процесса, и иконки должны показывать не только доступность действия (активен/неактивен), но и, например, время восстановления (Cooldown).
    
    \item \textbf{Мини-карта (Mini-map)}: В играх с открытым миром мини-карта предоставляет информацию о текущем местоположении игрока, а также о возможных целях или событиях вокруг. Она может быть интегрирована в интерфейс таким образом, чтобы отображать ключевые объекты или опасности в игре.
    
    \item \textbf{Модальные окна (Modal windows)}: В игре модальные окна могут быть использованы для отображения подсказок, сообщений об ошибках, уведомлений о достижениях или окон настроек, которые требуют от игрока принятия решения перед тем, как продолжить игру.
    
    \item \textbf{Жизнеизмерители и индикаторы состояния (Health and status bars)}: Эти элементы являются неотъемлемой частью игрового интерфейса, показывая текущее здоровье персонажа, уровни энергии и других ресурсов. Они могут менять цвет или форму в зависимости от критической ситуации (например, красная полоса здоровья, когда персонаж умирает).
\end{itemize}

\paragraph{Особенности кастомизации в игре}
\begin{itemize}
    \item \textbf{Изменение внешнего вида интерфейса}: Игрок может настроить цвета, шрифты или расположение элементов интерфейса для удобства.
    
    \item \textbf{Настройки управления}: Позволяет переназначать клавиши или изменить чувствительность контроллера для удобства игрового процесса.
    
    \item \textbf{Визуальные и аудио настройки}: Могут включать анимацию кнопок, дополнительные визуальные эффекты при активации навыков или изменение громкости звуковых эффектов.
\end{itemize}
\subsection{Графика и видео}

\subsubsection{Общее описание}
\paragraph{Техническое исполнение.}
Игра использует стилизованную \textbf{2D} графику с высоким разрешением, сочетающую в себе элементы \textit{pixel art} и современной векторной графики.  Высокое разрешение позволяет передать детализацию текстур и окружения, не теряя при этом характерной пиксельной стилизации, вдохновленной играми \textit{Katana Zero}, \textit{Hotline Miami}, и \textit{Cyberpunk 2077}.   Анимация плавная и выразительная, подчеркивая динамику игрового процесса.
\paragraph{Стилистика, атмосфера и палитра.}
Визуальный стиль игры —  \textbf{киберпанк},  передающий мрачную и напряженную атмосферу разрушенного города.  Палитра  основана на контрасте приглушенных темных и ярких оттенков.  Темные тона —  приглушенные фиолетовые (#ac61b9, #b7c1de) и синие (#0b468c, #092047) цвета,  серый металлик —  создают ощущение безысходности и клаустрофобии трущоб.  Яркие цвета, в основном розовые, синие и зеленые,  выделяют ключевые элементы окружения,  подчеркивая технологичность мира.  Доминируют текстуры ржавчины, изношенного металла, и потрескавшегося бетона.
\paragraph{Световые эффекты.}
Игра активно использует динамическое освещение и световые эффекты. Неоновые вывески,  фары автомобилей, и вспышки выстрелов создают  яркие, контрастные блики на темном фоне, подчеркивая детализацию и глубину сцен.  Тень играет важную роль в передаче атмосферы,  создавая ощущение тревоги и неопределенности.


\subsubsection{Двумерная графика и анимация}

Графика выполнена в пиксельном стиле, вдохновленном киберпанк-эстетикой, с использованием ярких неоновых цветов на фоне тёмных, мрачных тонов. Детализация ограничена размером пикселей, но акцент сделан на выразительных силуэтах, игре света и динамических эффектах.

\paragraph{Интерфейс.}
Интерфейс выполнен в ретрофутуристическом стиле с элементами, напоминающими старые цифровые панели:
\begin{itemize}
    \item \textbf{Панель состояния:} индикаторы здоровья и энергии представлены в виде неоновых полосок с эффектом "перезагрузки" при восстановлении.
    \item \textbf{Карта:} стилизована под пиксельный "голографический дисплей" с мерцающими пикселями.
    \item \textbf{Меню улучшений:} экран улучшений показывает пиксельные схемы имплантов с наглядной визуализацией их подключения.
    \item \textbf{Окна диалогов:} выполнены в стиле старых текстовых терминалов, с анимацией "появления текста" символ за символом.
\end{itemize}

\paragraph{Эффекты.}
Эффекты в пиксельной графике подчеркивают технологическую тематику:
\begin{itemize}
    \item \textbf{"Глитч-эффекты":} пиксельные искажения экрана, как будто данные повреждены.
    \item \textbf{Световые эффекты:} мигающие неоновые вывески, разряды энергии и вспышки от взаимодействий с объектами.
    \item \textbf{Анимация интерфейсов:} мерцание текста и случайные «цифровые сбои».
\end{itemize}

\paragraph{Основная игровая графика.}
Ключевые элементы игрового мира и персонажей проработаны с учётом ограничений пиксельной графики, при этом акцент сделан на передаче атмосферы.

\subparagraph{Персонажи.}
\begin{itemize}
    \item \textbf{Главный герой — «кибер-механик»:} образ включает множество ярких деталей, таких как светящиеся импланты, шлем с динамическим отображением данных и механические руки. Его движения сопровождаются легкими "электронными шлейфами".
    \item \textbf{Члены культа "Чистота":} строгий дизайн с темными силуэтами, символика культа выделяется ярко-красными акцентами. Они выглядят зловеще даже в пиксельной интерпретации.
    \item \textbf{Граждане:} разнообразные модификации тела, такие как протезы, антенны и визоры. Некоторые персонажи выглядят "искажёнными", чтобы подчеркнуть влияние технологий.
\end{itemize}

\subparagraph{Строения и юниты.}
\begin{itemize}
    \item \textbf{Кибернетические устройства:} дроны, турели и роботы, оформлены с высокой степенью стилизации: мерцающие лампочки, вращающиеся пиксельные детали.
    \item \textbf{Локации:} включают разнообразные футуристические здания, от высокотехнологичных башен до заброшенных складов с грудами старых имплантов.
\end{itemize}

\subparagraph{Игровой мир.}
\begin{itemize}
    \item \textbf{Ландшафты:} урбанистическая среда с высокой плотностью деталей: неоновые вывески, трубы с паром, дроны, парящие в небе. На фоне — темное звёздное небо или мрачная дождливая атмосфера.
    \item \textbf{Статические объекты:} терминалы, панели управления, разбитые экраны, разрушенные серверы и уличные киоски, стилизованные под футуристические автоматы.
\end{itemize}

\paragraph{Анимация.}
\begin{itemize}
    \item \textbf{Движения персонажей:} многокадровая анимация ходьбы, смерти, атаки и использования способностей. Эффекты имплантов дополняются пиксельным свечением.
    \item \textbf{Окружение:} мелькание неоновых вывесок, мигающие огни, дым.
    \item \textbf{Кат-сцены:} ключевые сюжетные моменты сопровождаются пиксельными анимациями, выполненными в общей стилистике игры.
\end{itemize}

\subsection{Звуки и музыка}
\subsubsection{Общее описание}
Аудио-составляющая игры создает напряженное и мрачное настроение, подчеркивая киберпанковскую атмосферу шутера в сочетании с элементами файтинга. Музыкальные темы отличаются динамичными ритмами и электронным звучанием, создавая ощущение безысходности и угнетенности в мире, пропитанном технологическим прогрессом и коррупцией. Звуковые эффекты помогают улучшить погружение в игровой процесс, от звуков выстрелов до приглушенных разговоров в затемненных уголках трущоб.

\subsubsection{Звук и звуковые эффекты}
\begin{itemize}
    \item \textbf{Интерфейс:} Звуки кнопок, уведомления о статусах, сигналы об успехах/неудачах, которые подчеркивают важные моменты во время игровой сессии.
    \item \textbf{Спецэффекты:} Шумы окружения, от отдаленных криков людей до единичных звуков выстрелов, создающих атмосферу беспокойства и опасности.
    \item \textbf{Игровые объекты:} Звуки взаимодействия с предметами, такие как хруст металла, когда главный герой подбирает оружие или активирует устройства; звук движения и шагов, которые отражают состояние персонажа.
    \item \textbf{Диалоги и анимационные вставки:} Запись голосов персонажей, озвучивание ключевых моментов и эмоций, помощь игроку в понимании мотивации и прошлого главного героя.
\end{itemize}

\subsubsection{Музыка}
\begin{itemize}
    \item \textbf{Главный интерфейс новой игры:}  Атмосферная электронная музыка,  использующая  глубокий  бас  и  сложные  мелодии,  создает  чувство  ожидания  и  напряженности,  подготавливая  игрока  к  погружению  в  мир  игры.
    \item \textbf{Музыкальное сопровождение роликов:}  Короткие,  но  эмоционально  насыщенные  треки,  написанные  в  стиле  темной  электроники,  подчеркивают  ключевые  моменты  сюжета  и  действия  персонажей,  усиливая  драматический  эффект.
    \item \textbf{Темы по уровням игры:}  Каждый  уровень  имеет  свою  уникальную  музыкальную  тему,  отражающую  особенности  местности,  атмосферу  и  сюжетные  события  данной  локации.  Это  позволяет  игроку  лучше  ориентироваться  в  игровом  мире  и  чувствовать  изменение  ситуации.  Например,  уровень  в  заброшенной  лаборатории  имеет  более  мрачную  и  тревожную  тему,  чем  уровень  в  ярком  и  многолюдном  районе  города.
    \item \textbf{Особенные фрагменты:}  Музыка  динамически  изменяется  в  зависимости  от  ситуации.  Напряженная  музыка  с  быстрым  темпом  и  резкими  переходами  сопровождает  сцены  действия,  в  то  время  как  более  спокойные  и  меланхоличные  ноты  звучат  в  моменты  потери  или  перемен.  Музыка  акцентирует  внимание  на  драматических  событиях,  подчеркивая  эмоциональную  насыщенность  сюжета.
\end{itemize}

\subsection{Описание уровней}
\subsubsection{Общее описание дизайна уровней}
Уровни игры  представляют  собой  сложно  продуманные  многоуровневые  локации,  наполненные  интерактивными  объектами,  тайными  проходами,  возможностями  для  исследования  и,  конечно же,  врагами.  Дизайн  каждого  уровня  уникален  и  связан  с  сюжетом  игры,  отражая  его  особенности  и  создавая  неповторимую  атмосферу.  В  игре  используется  нелинейный  дизайн  уровней,  позволяющий  игроку  выбирать  свой  путь  и  подход  к  решению  задач.  Уровни  содержат  различные  типы  среды,  от  темных  и  заброшенных  улиц  мегаполиса  до  высокотехнологичных  лабораторий  и  заводов.  Проработка  деталей  и  атмосферы  каждой  локации  является  одним  из  ключевых  аспектов  дизайна  уровней.

\subsubsection{Диаграмма взаимного расположения уровней}
Игра имеет линейно-последовательную структуру уровней, где каждый этап представляет собой отдельную локацию, связанную с сюжетными и игровыми задачами. Ниже приведён перечень основных локаций и их взаимосвязей:

\begin{itemize}
    \item \textbf{Центральный узел города:} начальная локация, где игрок знакомится с базовой механикой и основными NPC
    \item \textbf{Подземный комплекс:} первый вызов для игрока, представляющий собой лабиринт технических туннелей с охраной и ловушками
    \item \textbf{Кибернетическая фабрика:} место производства имплантов и важный сюжетный пункт, где игрок узнаёт больше о саботаже
    \item \textbf{Лаборатория культа "Чистота":} секретная база, где разрабатываются технологии для борьбы с имплантами
    \item \textbf{Главный серверный центр:} финальная локация, где игрок сталкивается с выбором между уничтожением ИИ или борьбой с культом
\end{itemize}

\noindent
Особые места:
\begin{itemize}
    \item \textbf{Неоновые улицы:} межлокационное пространство, которое игрок посещает для выполнения побочных заданий и взаимодействия с NPC
    \item \textbf{Скрытые зоны:} дополнительные необязательные области, содержащие редкие улучшения и фрагменты лора
\end{itemize}

\subsubsection{График введения новых объектов}

Развитие игры происходит через постепенное введение новых механик, врагов и улучшений для поддержания интереса и усложнения игрового процесса. График представлен следующим образом:

\begin{itemize}
    \item \textbf{Уровень 1: Центральный узел города}
    \begin{itemize}
        \item Базовые враги: простые дроны и патрули
        \item Основные механики: движение, атака, использование взлома
    \end{itemize}
    
    \item \textbf{Уровень 2: Подземный комплекс}
    \begin{itemize}
        \item Новые враги: ловушки, защищённые турели
        \item Новые механики: разминирование, скрытность
        \item Первое улучшение: усилитель энергии для имплантов
    \end{itemize}
    
    \item \textbf{Уровень 3: Кибернетическая фабрика}
    \begin{itemize}
        \item Новые враги: тяжёлые дроны с бронёй
        \item Новые механики: использование окружающих объектов для атаки
        \item Второе улучшение: способность временной невидимости
    \end{itemize}
    
    \item \textbf{Уровень 4: Лаборатория культа "Чистота"}
    \begin{itemize}
        \item Новые враги: фанатики культа с технологиями подавления имплантов
        \item Новые механики: необходимость отключения подавляющих устройств
        \item Третье улучшение: расширение навыков взлома
    \end{itemize}
    
    \item \textbf{Уровень 5: Главный серверный центр}
    \begin{itemize}
        \item Финальный вызов: сложные бои с культистами и высокоуровневыми дронами, финальный босс
    \end{itemize}
\end{itemize}

\section{Контакты}

E-mail: yusha@gmail.com

\newpage

\section*{Системные требования и платформы}

\end{document}

