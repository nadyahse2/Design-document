\documentclass{article}
\usepackage{graphicx} 
\usepackage[T2A]{fontenc}
\usepackage{newtxtext, newtxmath}
\usepackage[utf8]{inputenc}
\usepackage{geometry}
\geometry{top=2cm, bottom=2cm, left=2.5cm, right=2.5cm}
\usepackage{array}

\title{Дизайн документ}
\author{Надежда Мельникова, Валентина Дашинимаева, Наталья Хлопочкина, \\
Степан Васильченко, Никита Маряхин, Александр Фурман}
\date{Ноябрь 2024}

\begin{document}

\maketitle
\renewcommand{\contentsname}{Оглавление}
\tableofcontents

\section{Введение}

\section{Концепция}

\subsection{Введение}
\ \ \ \ \ В старом городе ты играешь за закаленного жизнью маргинала, который уже привык к жизни в трущобах и катакомбах. В мире, где импланты — обыденность, а замена конечностей — рядовая хирургическая процедура, твои механические руки — единственное, что осталось от прежней жизни. Проснувшись в жуткой хирургической палате с сильной головной болью и пустым местом, где прежде были твои протезы, ты понимаешь — тебя ограбили. Начинается твое путешествие в поисках ответа на вопрос: кто посмел лишить тебя самого ценного? И не только ответа, но и жестокой мести. Готовься к опасным приключениям, раскрытию темного прошлого героя и нелёгкому пути к справедливости в этом суровом мире.
\subsection{Жанр и аудитория}
\ \ \ \ \ Наша игра будет совмещать в себе несколько жанров: киберпанк, шутер и файтинг.

Игра направлена на людей, которым нравятся игры с  мрачной атмосферой и динамичными боевыми действиями с использованием разнообразного оружия и тактик. 

Возрастное ограничение: 18+
\subsection{Основные особенности игры}

\subsection{Описание игры}
\ \ \ \ \ Оказавшись на самом дне социума, герой давно привык к своей маргинальной
жизни в трущобах и катакомбах старого города, который был разгромлен после (пока не
уточняется) технической революции. Технологии настолько влились в повседневную жизнь
человека, что операция по замене частей тела на искусственные импланты стала самой
востребованной среди хирургов. Стоит такая процедура далеко не гроши, поэтому
единственное, что главный герой еще считает своей ценностью - его механические руки.

Завязка сюжета начинается в момент, когда главный герой обнаруживает себя в
странной мрачноватого вида хирургической палате, проснувшись с сильной болью в
голове и звоном в ушах. Вчерашний вечер он провел как обычно - напившись до
беспамятства. Он вообще привык просыпаться в незнакомых местах, однако в этот раз все
было серьезно - кто-то своровал его механические протезы.

Игроку еще предоставится возможность познакомиться с нашим героем поближе,
узнать его прошлое, а главное отомстить тому, кто обворовал его. 
\subsection{Предпосылки создания}
\ \ \ \ \ Игра опирается на актуальные тенденции рынка, где киберпанковая эстетика и глубокий нарратив продолжают привлекать внимание игроков. 
После успеха таких проектов, как Cyberpunk 2077, Deus Ex и Ghostrunner, очевидно, что жанр киберпанк сохраняет популярность.
Игра стремится предложить уникальный взгляд на жанр, акцентируя внимание на психологических аспектах и социальном неравенстве, 
что делает её интересной для широкой аудитории.

Рынок также демонстрирует рост популярности игр с динамичными механиками боя и нелинейным повествованием. По данным аналитических отчётов 
в 2023 году игры с подобными характеристиками показывали устойчивый рост продаж, особенно на платформах ПК и консолях. Это подтверждает, 
что игра имеет потенциал занять достойное место среди успешных проектов.

Вопросы лицензирования в разработке были заранее учтены. Мы используем лицензированный движок Unreal Engine, что обеспечивает 
высокую производительность и визуальную привлекательность игры. Кроме того, весь музыкальный и графический контент либо создаётся 
собственными силами, либо приобретается через лицензированные платформы (например, Unity Asset Store или AudioJungle), что гарантирует 
отсутствие проблем с авторскими правами.
\subsection{Платформа}

\section{Функциональная спецификация}

\subsection{Принципы игры}

\subsubsection{Суть игрового процесса}

\subsubsection{Ход игры и сюжет}

\subsection{Физическая модель}

\subsection{Персонаж игрока}

\subsection{Элементы игры}

\subsection{«Искусственный интеллект»}

\subsection{Многопользовательский режим}

\subsection{Интерфейс пользователя}

\subsubsection{Блок-схема}

\subsubsection{Функциональное описание и управление}

\subsubsection{Объекты интерфейса пользователя}

\subsection{Графика и видео}

\subsubsection{Общее описание}

\subsubsection{Двумерная графика и анимация}

\subsubsection{Трехмерная графика и анимация}

\subsubsection{Анимационные вставки}

\subsection{Звуки и музыка}

\subsubsection{Общее описание}

\subsubsection{Звук и звуковые эффекты}

\subsubsection{Музыка}

\subsection{Описание уровней}

\subsubsection{Общее описание дизайна уровней}

\subsubsection{Диаграмма взаимного расположения уровней}

\subsubsection{График введения новых объектов}

\section{Контакты}

\newpage

\section*{Системные требования и платформы}

Игра разрабатывается для следующих платформ:
\begin{itemize}
    \item PC (Windows, macOS, Linux)
    \item PlayStation
    \item Xbox
    \item Nintendo Switch
\end{itemize}

Для PC-версии игры минимальные и рекомендуемые системные требования приведены ниже:

\begin{table}[h!]
\centering
\renewcommand{\arraystretch}{1.5}
\begin{tabular}{|l|l|l|}
\hline
\textbf{Требования}           & \textbf{Минимальные}                        & \textbf{Рекомендуемые}                    \\ \hline
Операционная система          & Windows 10, Ubuntu 20.04                    & Windows 11, macOS Ventura                 \\ \hline
Процессор                     & Intel Core i3-6100 / AMD FX-6300            & Intel Core i5-10400 / AMD Ryzen 5 3600    \\ \hline
ОЗУ                           & 8 ГБ                                        & 16 ГБ                                     \\ \hline
CD-ROM привод                 & Не требуется                               & Не требуется                             \\ \hline
Свободное место на HDD        & 20 ГБ                                       & 50 ГБ (SSD рекомендуется)                \\ \hline
Видеокарта                    & NVIDIA GTX 750 Ti / AMD R7 260X             & NVIDIA GTX 1660 / AMD RX 580             \\ \hline
Звуковая карта                & Любая совместимая с DirectX                 & Высококачественная                       \\ \hline
Управление                    & Клавиатура и мышь                           & Геймпад (рекомендуется)                  \\ \hline
\end{tabular}
\caption{Системные требования для PC}
\end{table}

Дополнительное оборудование: Не требуется.

\end{document}

