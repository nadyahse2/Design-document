\documentclass{article}
\usepackage{graphicx} 
\usepackage[T2A]{fontenc}
\usepackage{newtxtext, newtxmath}
\usepackage[utf8]{inputenc}
\usepackage{geometry}
\geometry{top=2cm, bottom=2cm, left=2.5cm, right=2.5cm}
\usepackage{array}

\title{Дизайн документ}
\author{Надежда Мельникова, Валентина Дашинимаева, Наталья Хлопочкина, \\
Степан Васильченко, Никита Маряхин, Александр Фурман}
\date{Ноябрь 2024}

\begin{document}

\maketitle
\renewcommand{\contentsname}{Оглавление}
\tableofcontents

\section{Введение}

\ \ \ \ \ Здесь представлена информация, необходимая для понимания данного документа, посвященного разработке игры в жанре киберпанк, шутер и файтинг, с возрастным ограничением 16+ / 18+.

\subsection{Комментарии по организации содержимого документа}
\ \ \ \ \ Документ разделен на несколько ключевых разделов, включая концепцию игры, персонажей, игровой мир, механики и стилистику. Каждый раздел подробно описывает центральные элементы игры и будет служить основой для дальнейшей разработки.


\subsection{Список авторов}
\begin{itemize}
    \item Надежда Мельникова 
    \item Валентина Дашинимаева 
    \item Наталья Хлопочкина
    \item Степан Васильченко 
    \item Никита Маряхин 
    \item Александр Фурман 
\end{itemize}


\subsection{Дополнительные сведения}
\ \ \ \ \ Данный документ предназначен для команды разработки и заинтересованных сторон. Знание контекста игры, ее тематики и целевой аудитории является важным для обсуждений и успешной реализации идеи. Игра устанавливает возрастные ограничения 16+ / 18+, что подразумевает, что все элементы, такие как сюжетная линия, визуальные и аудио материалы, должны полностью соответствовать этим критериям.

\section{Концепция}

\subsection{Введение}
\ \ \ \ \ В старом городе ты играешь за закаленного жизнью маргинала, который уже привык к жизни в трущобах и катакомбах. В мире, где импланты — обыденность, а замена конечностей — рядовая хирургическая процедура, твои механические руки — единственное, что осталось от прежней жизни. Проснувшись в жуткой хирургической палате с сильной головной болью и пустым местом, где прежде были твои протезы, ты понимаешь — тебя ограбили. Начинается твое путешествие в поисках ответа на вопрос: кто посмел лишить тебя самого ценного? И не только ответа, но и жестокой мести. Готовься к опасным приключениям, раскрытию темного прошлого героя и нелёгкому пути к справедливости в этом суровом мире.
\subsection{Жанр и аудитория}
\ \ \ \ \ Наша игра будет совмещать в себе несколько жанров: киберпанк, шутер, файтинг, платформер и Hack and slash.

Игра направлена на людей, которым нравятся игры с  мрачной атмосферой и динамичными боевыми действиями с использованием разнообразного оружия и тактик. 

Возрастное ограничение: 18+
\subsection{Основные особенности игры}

\subsection{Описание игры}
\ \ \ \ \ Погрузитесь в мрачный и детально проработанный мир киберпанка, где технологии переплетаются с повседневной жизнью до неузнаваемости. Ваша история начинается в разрушенном старом городе, пострадавшем от масштабной технической революции. Главный герой — житель трущоб и катакомб, привыкший к маргинальной жизни на самом дне социума. В этом мире операции по замене частей тела на искусственные импланты стали нормой, но дорогостоящей роскошью. Единственная ценность, которой по-прежнему дорожит герой — его механические руки.

\subsubsection{Сюжетная линия}

\ \ \ \ \ История начинается с неожиданного пробуждения главного героя в странной и мрачной хирургической палате. Голова раскалывает боль, а уши звонят после вчерашнего вечера, проведенного в запое. Обычно герой не удивляется, просыпаясь в незнакомых местах, но на этот раз что-то не так: его ценные механические протезы исчезли. Теперь перед ним стоит задача не только выжить, но и отомстить тем, кто лишил его части тела и, возможно, личности.

\subsubsection{Игровой процесс}

\ \ \ \ \ Игроку предстоит управлять героем, используя сочетание традиционного огнестрельного оружия и боевых навыков, а также доступных имплантов для преодоления препятствий и сражений. Игра сочетает элементы шутера и файтинга, предоставляя динамичные и напряженные бои с разнообразными противниками. В процессе прохождения игрок будет:
 • Исследовать мрачные районы города: От заброшенных трущоб до высокотехнологичных комплексов, каждый район полон секретов и опасностей;
 • Бороться с врагами: Использовать различные виды оружия и боевых приемов, а также улучшать свои импланты для повышения боевых возможностей;
 • Развивать персонажа: Собирая ресурсы и выполняя задания, игрок сможет улучшать механические руки и приобретать новые способности;
 • Взаимодействовать с NPC: Узнавать больше о прошлом героя, заводить союзников и противостоять врагам, раскрывая глубокие сюжетные линии;
 • Принимать решения: Ваши выборы будут влиять на развитие сюжета и концовку игры, создавая уникальный опыт прохождения.

\subsubsection{Визуальный и звуковой стиль}

\ \ \ \ \ Игра вдохновлена такими проектами, как Katana Zero, Hotline Miami и Cyberpunk 2077. Визуальный стиль сочетает неоновую палитру киберпанка с мрачными, детализированными локациями, создавая атмосферу напряженности и неизведанности. Саундтрек включает синтезаторные композиции и динамичные треки, усиливающие эмоциональное воздействие и погружение в игровой мир.




\subsection{Предпосылки создания}
\ \ \ \ \ Игра опирается на актуальные тенденции рынка, где киберпанковая эстетика и глубокий нарратив продолжают привлекать внимание игроков. 
После успеха таких проектов, как Cyberpunk 2077, Deus Ex и Ghostrunner, очевидно, что жанр киберпанк сохраняет популярность.
Игра стремится предложить уникальный взгляд на жанр, акцентируя внимание на психологических аспектах и социальном неравенстве, 
что делает её интересной для широкой аудитории.

Рынок также демонстрирует рост популярности игр с динамичными механиками боя и нелинейным повествованием. По данным аналитических отчётов 
в 2023 году игры с подобными характеристиками показывали устойчивый рост продаж, особенно на платформах ПК и консолях. Это подтверждает, 
что игра имеет потенциал занять достойное место среди успешных проектов.

Вопросы лицензирования в разработке были заранее учтены. Мы используем лицензированный движок Unreal Engine, что обеспечивает 
высокую производительность и визуальную привлекательность игры. Кроме того, весь музыкальный и графический контент либо создаётся 
собственными силами, либо приобретается через лицензированные платформы (например, Unity Asset Store или AudioJungle), что гарантирует 
отсутствие проблем с авторскими правами.
\subsection{Платформа}

\section{Функциональная спецификация}

\subsection{Принципы игры}

\subsubsection{Суть игрового процесса}

\subsubsection{Ход игры и сюжет}

\subsection{Физическая модель}

\subsection{Персонаж игрока}

\subsection{Элементы игры}

\subsection{«Искусственный интеллект»}

\subsection{Многопользовательский режим}

\subsection{Интерфейс пользователя}

\subsubsection{Блок-схема}

\subsubsection{Функциональное описание и управление}

\subsubsection{Объекты интерфейса пользователя}

\subsection{Графика и видео}

\subsubsection{Общее описание}

\subsubsection{Двумерная графика и анимация}

\subsubsection{Трехмерная графика и анимация}

\subsubsection{Анимационные вставки}

\subsection{Звуки и музыка}

\subsubsection{Общее описание}

\subsubsection{Звук и звуковые эффекты}

\subsubsection{Музыка}

\subsection{Описание уровней}

\subsubsection{Общее описание дизайна уровней}

\subsubsection{Диаграмма взаимного расположения уровней}

\subsubsection{График введения новых объектов}

\section{Контакты}

\newpage

\section*{Системные требования и платформы}

Игра разрабатывается для следующих платформ:
\begin{itemize}
    \item PC (Windows, macOS, Linux)
    \item PlayStation
    \item Xbox
    \item Nintendo Switch
\end{itemize}

Для PC-версии игры минимальные и рекомендуемые системные требования приведены ниже:

\begin{table}[h!]
\centering
\renewcommand{\arraystretch}{1.5}
\begin{tabular}{|l|l|l|}
\hline
\textbf{Требования}           & \textbf{Минимальные}                        & \textbf{Рекомендуемые}                    \\ \hline
Операционная система          & Windows 10, Ubuntu 20.04                    & Windows 11, macOS Ventura                 \\ \hline
Процессор                     & Intel Core i3-6100 / AMD FX-6300            & Intel Core i5-10400 / AMD Ryzen 5 3600    \\ \hline
ОЗУ                           & 8 ГБ                                        & 16 ГБ                                     \\ \hline
CD-ROM привод                 & Не требуется                               & Не требуется                             \\ \hline
Свободное место на HDD        & 20 ГБ                                       & 50 ГБ (SSD рекомендуется)                \\ \hline
Видеокарта                    & NVIDIA GTX 750 Ti / AMD R7 260X             & NVIDIA GTX 1660 / AMD RX 580             \\ \hline
Звуковая карта                & Любая совместимая с DirectX                 & Высококачественная                       \\ \hline
Управление                    & Клавиатура и мышь                           & Геймпад (рекомендуется)                  \\ \hline
\end{tabular}
\caption{Системные требования для PC}
\end{table}

Дополнительное оборудование: Не требуется.

\end{document}

